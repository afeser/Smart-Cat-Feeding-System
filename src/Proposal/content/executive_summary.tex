\section{Executive Summary} \label{sec:exec_sum}

Current methods of cat feeding give way to a few problems that are promised to be solved by Felerest. The proposed autonomous smart cat feeding system, by removing these problems, makes the lives of both humans and cats easier. The system, by identifying and recognizing cats, will feed them autonomously, keeping their weight in mind. The diverse, qualified project team is confident and ready to bring this project to life. Further descriptions of the problem, solution procedure, the team qualifications, and the deliverables follow in the next paragraphs of this summary, leaving the details to the corresponding sections.

Naturally, every single cat has its own dietary habits: some cats inevitably eat more than the others. This is the reason for two of the problems of current cat feeding methods/systems: leftover cat food and obese cats. Since the proposed system recognizes cats, it is able to feed them in such a way that cats are neither hungry nor overfed afterwards; the system clearly deals with the leftover and obesity problems.

The autonomy of the system comes in to play in situations where the cat owners forget to feed their cats or have to feed them several times a day with fresh food. There is also the problem of stray dogs disturbing eating cats. This problem is anticipated to be solved by a dog deterring sub module in the system.

The project team has five members, each interested and experienced in different areas that make up this project. The various sub modules of the system are each handled by members that are experienced in that particular area. Also, the teamwork orientation of the members ensure a seamless work environment.

Felerest aims to deliver the reports, the product, the web/mobile application and a user manual for these in a time span of about four months. The research and development cost for this project will be less than \$200.