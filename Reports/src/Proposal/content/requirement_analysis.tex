\section{Requirement Analysis} \label{sec:req_anal}

% TODO: 
% 1) Ekleme çıkarma yapılabilir.
% 3) Done... objektive tree de puanlar değişebilir değerleri salladım. 

Objectives of the project are feeding only cats, identifying new cats and recognizing them later, deterring dogs from the area, consuming low power, as well as being cost efficient, rechargeable, portable, easy to use and robust to environmental changes.
Also, feeding regime of identified cats are logged by the system. The status of the food supply and battery, profiles of the cats and feeding logs will be shared in the mobile/web application. The most important objective is the speed of the system since cats might be bored in few seconds.

Weighted objective tree can be seen in appendix A.

Requirements of the project are listed below.
\begin{itemize}
\item The non-removable, rechargeable Lithium-ion batteries will be completely charged from the grid in 3 hours.
\item Battery will last at least five hours. The equipment will run on 18650 Lithium-ion batteries.
\item There are no attachments to the animals. The animals will be identified in 4 seconds and their feeds will be automatically delivered. 
\item System should be lightweight in order to be easily carried by a single person. The system will be around 7 kilograms when the reservoir is full.
\item To avoid food waste, each cat will be fed three to five times in a day.
\item The dogs will be deterred with high frequency sounds in order not to damage them. The frequency will be around 25 to 30 kHz. However, high frequency sounds also affect cats. Solutions for this issue are being researched.
\item Capacity of the reservoir will be 5 kilograms which corresponds to around 55 meals.
\item Power usage will be minimized. Low current driven motor will be used.
\item A light bulb will be used at nights. When the motion sensor is activated the bulb will turn on, and it will turn off in 2 minutes. This way, redundant use of electricity will be eliminated.
\end{itemize}


