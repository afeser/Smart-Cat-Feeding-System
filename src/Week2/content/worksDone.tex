\section{Works Done}
\label{sec:worksDone}
This week's work was related to proposal report writing, chart drawing and project plan construction. In addition to these works, some of us made progress in some practical applications as well. 
\subsection{Mechanics}
This week, some other mechanical designs were introduced. One of them is using a gear system to operate the door. A simple drawing demonstrationg the design can be seen in Fig \ref{fig:mechDoga1}.

\begin{figure}
    \centering
    \includegraphics{img/mechDoga1.png}
    \caption{Door Design}
    \label{fig:mechDoga1}
\end{figure}

By using at least two gears, we can move the door back and forth. However, there exists the problem of cats blocking the door as it moves to close for more food. This problem can be fixed by making a thin slit around the door. By this way, the door cannot fall or lost its connection with the gear. Moreover, since the slits are narrow, cat food cannot get into the slits and block the door. Although it blocks the door somehow, the system will come to the equilibrium since other cat foods also jam between the door and the slit which will stop the food flow eventually. 
The main problem of this design is the mass of the foods since when the door is closed the considerable amount of food will be on the door and it will create a force on the door. Therefore, the door will get into contact with the lower part of the slit and this will create friction between the door and the slit. One has to apply high torque in order to move the door. The problem starts here. Our motor might not be enough to create that torque. Although there exist some motor that can deal with that force, we don’t prefer it since it will have high power consumption and its price will be high compared with low power motors.
In this week, we also have attempted to try the design of our old design, which was the salt cup idea. Cardboard was cut according to the design specifications but still, the tests aren't done. It is planned to be done in the next week. According to the result of the test, other designs might gain importance.


\subsection{Proposal Report}
To write proposal report, we create a work division among team. The aim was to distribute the work such that everyone will try to expert in his/her own area and make the best. Therefore, we developed work packages and shared based on the skills.

We also conduct research on how to create a Gantt chart \cite{cite:ganttchart}.
More information on work distribution and personal responsibilities are given in section \ref{sec:workDistribution}. The works are given in figure \ref{fig:ganttChart} as Gantt charts. The Gantt chart is created in LibreOffice.

\begin{figure}
    \centering
    \includegraphics[width=\linewidth]{img/proposalGanttChart.png}
    \caption{Proposal Gantt Chart}
    \label{fig:ganttChart}
\end{figure}

We realized that solution procedure part may be misunderstood which is why we contacted with our supervisor. However at the moment the report is being written, we do not know the correct way to improve it, therefore it is left to the next week to edit this part.
